\documentclass[12pt]{article}
\usepackage[a4paper,margin=1in]{geometry}
\usepackage{titlesec}
\usepackage{fancyhdr}
\usepackage{booktabs}
\usepackage{array}
\usepackage[table,xcdraw]{xcolor}
\usepackage[numbers]{natbib} % Adăugat [numbers] pentru stil numeric
\usepackage{hyperref}

\setlength{\headheight}{14.5pt}
\pagestyle{fancy}
\fancyhf{}
\rfoot{\thepage}

\titleformat{\title}{\normalfont\bfseries\LARGE}{\thesection}{1em}{}
\titleformat{\section}{\normalfont\large\bfseries}{\thesection}{1em}{}
\titleformat{\subsection}{\normalfont\normalsize\bfseries}{\thesubsection}{1em}{}

\title{\textbf{Hybrid Work Contracts in the Age of Artificial Intelligence}\\[0.5em]
\large A legal and economic transition model}
\author{Teodor Berger}
\date{May 2025}

\begin{document}
\sloppy % Previne depășirile în paragrafe
\maketitle

\section*{Abstract}
This project proposes a legal and economic model for hybrid work contracts in the age of artificial intelligence (AI). The model ensures job security, fair income distribution, and state responsibility in protecting employees affected by AI integration. It bridges the transitional gap caused by automation, preventing unjust layoffs and offering flexible schedules with age-based leave benefits. Additionally, a full-time non-hybrid model is introduced for employees working 40 hours/week in shifts, supported indirectly by AI efficiency, with a 12\% salary supplement and negotiable leave.

\section{Introduction}
The rapid adoption of AI in the labor market has led to significant job displacement. Governments and corporations often failed to anticipate the social consequences, resulting in premature layoffs. This model establishes protective measures for employees, ensuring no unjust dismissals occur due to AI integration. It introduces hybrid contracts combining AI systems and human workers, alongside a full-time non-hybrid model for employees in high-intensity roles, supported by AI-driven productivity gains \cite{brynjolfsson2023,acemoglu2021,arntz2016}.

\section{Objectives}
\begin{itemize}
  \item Prevent unjust layoffs post-AI integration.
  \item Establish fair salary percentages and leave benefits for hybrid and full-time contracts.
  \item Ensure state participation in income support with automatic cost recovery.
  \item Recognize age-based bonuses and leave to reward long-term service.
\end{itemize}

\section{Model Structure}

\subsection{Legal Principles}
\begin{enumerate}
  \item \textbf{No Unjust Dismissals:} Employees dismissed before AI integration must be protected or compensated unless layoffs were due to misconduct or company bankruptcy.
  \item \textbf{State Responsibility:} The state contributes 10\% of the salary cost in hybrid contracts, recovered automatically through increased corporate taxes from AI-driven productivity.
  \item \textbf{Contractual Duty:} AI cannot replace contractual obligations without renegotiation or employee consent.
\end{enumerate}

\subsection{Hybrid Work Compensation Model}
The hybrid model offers flexible schedules (1--5 days/week) with guaranteed salary percentages, state support, and age-based leave.

\rowcolors{1}{blue!10}{white}
\begin{table}[!htbp] % Ajustat float specifier
\centering
\begin{tabular}{|>{\bfseries}p{2.5cm}|p{2.5cm}|p{2.5cm}|p{2.5cm}|p{2.5cm}|} % Ajustat lățimea coloanelor
\hline
Workdays/Week & Weekly Hours & Base Salary (\%) & State Support (\%) & Leave (Days/Month) \\
\hline
1 & 3 & 55 & 10 & 2--3 \\
2 & 6 & 60 & 10 & 2--3 \\
3 & 9 & 65 & 10 & 2--3 \\
4 & 12 & 70 & 10 & 2--3 \\
5 & 15 & 75 & 10 & 2--3 \\
\hline
\end{tabular}
\caption{Hybrid Work Compensation Model with Monthly Leave}
\end{table}

\subsection{Age-Based Bonuses and Leave}
Bonuses and leave are provided based on age, with flexibility for industry-specific negotiations.

\begin{table}[!htbp]
\centering
\begin{tabular}{|>{\bfseries}p{3cm}|p{3cm}|p{3cm}|}
\hline
Age Range & Salary Bonus (\%) & Leave (Days/Month) \\
\hline
18--35 & 0 & 2 \\
36--54 & 4 & 2.5 \\
55+ & 7--10 & 3 \\
\hline
\end{tabular}
\caption{Age-Based Bonuses and Monthly Leave}
\end{table}

\subsection{Full-Time Non-Hybrid Model}
Employees working 40 hours/week in 3--4 shifts, supported indirectly by AI efficiency, receive a 12\% salary supplement and negotiable leave.

\begin{table}[!htbp]
\centering
\begin{tabular}{|>{\bfseries}p{3cm}|p{3cm}|p{3cm}|p{3cm}|}
\hline
Work Schedule & Weekly Hours & Salary Supplement (\%) & Leave (Days/Month) \\
\hline
3--4 Shifts & 40 & 12 & 2.5--4 \\
\hline
\end{tabular}
\caption{Full-Time Non-Hybrid Model with AI Support}
\end{table}

\subsection{Example Case: John, Age 27 (Hybrid Contract)}
\begin{itemize}
  \item \textbf{Work}: 4 days/week, 12 hours total
  \item \textbf{Base Salary}: 70\% (2100 EUR, from 3000 EUR base)
  \item \textbf{State Support}: +10\% (300 EUR)
  \item \textbf{Age Bonus}: 0\% (18--35 years)
  \item \textbf{Total Salary}: 2400 EUR
  \item \textbf{Leave}: 2 days/month
\end{itemize}

\subsection{Example Case: Sarah, Age 55+ (Full-Time Non-Hybrid)}
\begin{itemize}
  \item \textbf{Work}: 40 hours/week, 3 shifts
  \item \textbf{Base Salary}: 100\% (3000 EUR)
  \item \textbf{Shift Supplement}: +12\% (360 EUR)
  \item \textbf{Age Bonus}: +7\% (210 EUR)
  \item \textbf{Total Salary}: 3570 EUR
  \item \textbf{Leave}: 3.5 days/month (3 base + 0.5 negotiated)
\end{itemize}

\section{Public Cost Recovery and Employment Stability}

\subsection{Overview of the Tripartite Model}
The model ensures a balanced transition to AI-assisted workflows through cooperation between employees, employers, and the state, with no actor incurring losses.

\subsection{For Employees}
\begin{itemize}
  \item \textbf{Job Security:} All employees are transitioned to hybrid or full-time contracts, preventing layoffs due to AI.
  \item \textbf{Flexible Schedules:} Hybrid workers choose 1--5 days/week; full-time workers receive a 12\% salary supplement for shifts.
  \item \textbf{Leave Benefits:} 2--3 days/month for hybrid workers, 2.5--4 days for full-time, negotiable by industry.
  \item \textbf{Age Bonuses:} 4--10\% salary bonuses for employees aged 36+.
\end{itemize}

\subsection{For Employers}
\begin{itemize}
  \item \textbf{No Severance Costs:} Employees remain in the system, eliminating severance expenses.
  \item \textbf{Workforce Continuity:} Institutional knowledge is preserved.
  \item \textbf{State Support:} 10\% salary contribution for hybrid contracts, recovered via taxes on increased profits.
  \item \textbf{AI Freedom:} Firms can deploy AI, provided they retain staff under hybrid or full-time frameworks.
\end{itemize}

\subsection{For the State}
\begin{itemize}
  \item \textbf{Cost Recovery:} The 10\% contribution is recovered through higher corporate taxes from AI-driven productivity (e.g., 15--20\% profit increase).
  \item \textbf{Fiscal Revenue:} Increased productivity boosts taxable income.
  \item \textbf{Social Stability:} Maintaining employment reduces social tension and inequality.
  \item \textbf{Phased Withdrawal:} State contribution reduces from 10\% to 5\% over 24 months.
\end{itemize}

\subsection{Cost Recovery Mechanism}
The state's contribution is offset by:
\begin{itemize}
  \item \textbf{Higher Corporate Taxes:} AI-driven productivity increases profits, generating tax revenue.
  \item \textbf{Reduced Welfare Spending:} No unemployment benefits or retraining programs needed.
  \item \textbf{Phased Withdrawal:} State support decreases as companies stabilize.
\end{itemize}

\subsection{Legal Guarantees}
\begin{itemize}
  \item Hybrid and full-time contracts are mandated via national/EU legislation.
  \item Companies cannot terminate employment due to automation unless misconduct occurs.
  \item All roles are preserved and adapted during the transition.
\end{itemize}

\section{Implementation in the European Union}
The model aligns with EU objectives for a just digital transition:
\begin{itemize}
  \item \textbf{EU Directive:} A directive can mandate hybrid and full-time contracts, with state contributions (10\%) funded by EU funds (e.g., Recovery and Resilience Facility).
  \item \textbf{Pilot Programs:} Test in countries like the Netherlands (flexible labor markets) and Romania (IT and automotive sectors).
  \item \textbf{Monitoring:} The European Labour Authority (ELA) oversees implementation and fiscal impact.
\end{itemize}

\section{Conclusion}
This model defends labor rights in the AI era, ensuring no unjust layoffs, fair compensation, and age-based leave benefits. It supports both hybrid and full-time workers, with state contributions recovered automatically through increased tax revenue, ensuring no losses for employees, employers, or the state.

\begin{thebibliography}{9}
\bibitem{brynjolfsson2023}
Brynjolfsson, Erik and McAfee, Andrew (2023). \textit{Artificial Intelligence and the Future of Work}. Journal of Economic Perspectives, 37(2), 25--48.

\bibitem{acemoglu2021}
Acemoglu, Daron and Restrepo, Pascual (2021). \textit{AI and Jobs: Evidence from Online Vacancies}. NBER Working Paper No. 28961.

\bibitem{arntz2016}
Arntz, Melanie, Gregory, Terry, and Zierahn, Ulrich (2016). \textit{The Risk of Automation for Jobs in OECD Countries}. OECD Social, Employment and Migration Working Papers, 189.
\end{thebibliography}

\end{document}
