\documentclass[12pt]{article}
\usepackage[a4paper,margin=1in]{geometry}
\usepackage{titlesec}
\usepackage{fancyhdr}
\usepackage{graphicx}
\usepackage{booktabs}
\usepackage{array}
\usepackage[table,xcdraw]{xcolor}
\usepackage{lipsum}
\usepackage{natbib}
\usepackage{hyperref}

\setlength{\headheight}{14.5pt}
\pagestyle{fancy}
\fancyhf{}
\rfoot{\thepage}

\titleformat{\title}{\normalfont\bfseries\LARGE}{\thesection}{1em}{}
\titleformat{\section}{\normalfont\large\bfseries}{\thesection}{1em}{}
\titleformat{\subsection}{\normalfont\normalsize\bfseries}{\thesubsection}{1em}{}

\title{\textbf{Hybrid Work Contracts in the Age of Artificial Intelligence}\\[0.5em]
\large A legal and economic transition model}
\author{Teodor Berger}
\date{May 2025}

\begin{document}
\sloppy
\maketitle

\section*{Abstract}
This project proposes a legal and economic model to support hybrid work contracts in the age of artificial intelligence. The model addresses job security, fair income distribution, and state responsibility in protecting employees affected by AI integration. It aims to bridge the transitional gap caused by the sudden automation of human labor before governments had time to legislate or adapt.

\section{Introduction}
The advent of AI in the labor market has triggered widespread replacement of human jobs. Governments and corporations failed to anticipate the social consequences, leading to premature layoffs. This model aims to establish protective measures for employees who were dismissed before the introduction of hybrid contracts involving both AI systems and human workers.

\section{Objectives}
\begin{itemize}
  \item Prevent unjust layoffs post-AI integration.
  \item Establish fair salary percentages for hybrid contracts.
  \item Ensure state participation in income support.
  \item Recognize age-based bonuses as compensation for long-term service.
\end{itemize}

\section{Model Structure}

\subsection{Legal Principles}
\begin{enumerate}
  \item \textbf{No Unjust Dismissals:} Employees dismissed before AI integration must be protected or compensated unless layoffs were due to misconduct or company bankruptcy.
  \item \textbf{State Responsibility:} The state shares 10\% of the salary cost in hybrid contracts to stabilize income.
  \item \textbf{Contractual Duty:} AI cannot be a legal substitute for contractual obligations without renegotiation or consent.
\end{enumerate}

\subsection{Weekly Structure and Compensation}

\rowcolors{1}{blue!10}{white}
\begin{table}[h!]
\centering
\begin{tabular}{>{\bfseries}c c c c c}
Workdays/week & Weekly Hours & Base Salary (\%) & State Support (\%) & Total (\%) \\
\midrule
1 & 3 h & 55 & 10 & 65 \\
2 & 6 h & 60 & 10 & 70 \\
3 & 9 h & 65 & 10 & 75 \\
4 & 12 h & 70 & 10 & 80 \\
5 & 15 h & 75 & 10 & 85 \\
\end{tabular}
\caption{Hybrid Work Compensation Model}
\end{table}

\subsection{Age-Based Bonuses}

\begin{table}[h!]
\centering
\begin{tabular}{>{\bfseries}c c}
Age Range & Bonus (\%) \\
\midrule
50--55 & 4 \\
56--60 & 7 \\
61+ & 10 \\
\end{tabular}
\caption{Seniority Bonus}
\end{table}

\subsection{Example Case: Andrei, Age 57}

\begin{itemize}
  \item Work: 4 days/week, 12h total
  \item Base salary: 70\%
  \item State support: +10\%
  \item Age bonus: +7\%
  \item \textbf{Total: 87\%}
\end{itemize}

\section{Public Cost Recovery and Employment Stability}

\subsection{Overview of the Tripartite Model}

The hybrid work contract model is designed as a tripartite cooperation between the employee, the employer, and the state. Each actor contributes to a stable transition to AI-assisted workflows, with distinct responsibilities and reciprocal benefits.

\subsection{For Employees}

\begin{itemize}
  \item \textbf{Job security:} All current employees are transitioned into hybrid contracts by default, ensuring no layoffs occur due to AI implementation.
  \item \textbf{Flexible schedules:} Employees can choose a reduced work week (1–5 days), with salaries adjusted proportionally but protected by a guaranteed minimum coverage of 55\% to 75\% plus bonuses.
  \item \textbf{Age bonuses:} Older employees receive salary supplements (4\% for 50–55 years, 7\% for 56–60, and 10\% for 61+), recognizing long-term service.
  \item \textbf{Dignity preserved:} Employees remain valuable members of the labor force and are not discarded due to technological progress.
\end{itemize}

\subsection{For Companies}

\begin{itemize}
  \item \textbf{No severance costs:} Companies are not required to pay severance packages, as employees remain within the system.
  \item \textbf{Workforce continuity:} Institutional knowledge and team stability are preserved.
  \item \textbf{State support:} The state covers 10\% of each adjusted hybrid salary, reducing labor costs during the transition period.
  \item \textbf{AI implementation freedom:} Firms can deploy AI systems to assist tasks, provided they retain staff under the hybrid framework.
\end{itemize}

\subsection{For the State}

\begin{itemize}
  \item \textbf{Cost-efficient transition:} The 10\% public contribution is far lower than the long-term cost of unemployment benefits, retraining programs, or economic stagnation caused by mass layoffs.
  \item \textbf{Increased fiscal revenue:} With improved productivity and AI-supported efficiency, companies generate higher taxable income and corporate tax.
  \item \textbf{Social stability:} Maintaining employment rates reduces social tension and inequality.
  \item \textbf{Automatic recovery:} As companies become more efficient, the state's contribution can gradually phase out (e.g., over 24 months).
\end{itemize}

\subsection{Cost Recovery Mechanism}

The state's temporary support is offset by:
\begin{itemize}
  \item \textbf{Higher corporate tax revenue} due to AI-augmented productivity.
  \item \textbf{Reduced welfare spending}—no unemployment benefits or subsidized retraining programs are needed.
  \item \textbf{Phased withdrawal} of the 10\% over time as AI becomes integrated and companies stabilize revenue margins.
\end{itemize}

\subsection{Legal Guarantees}

\begin{itemize}
  \item Hybrid contracts are introduced via national legislation and automatically apply to all active employees at the time of adoption.
  \item Companies may not terminate employment due to task automation unless specific misconduct occurs.
  \item All roles must be preserved and adapted; job elimination is prohibited during the transition period.
\end{itemize}

\section{Conclusion}
This model defends labor rights in the AI age. Those who were dismissed before governments regulated AI integration must be compensated or reinstated via protected hybrid contracts. Technology should not overwrite dignity, fairness, or human contribution.

\section*{References}

\begin{itemize}
\item Brynjolfsson, Erik and McAfee, Andrew (2023). \textit{Artificial Intelligence and the Future of Work}. Journal of Economic Perspectives, 37(2), 25--48.
\item Acemoglu, Daron and Restrepo, Pascual (2021). \textit{AI and Jobs: Evidence from Online Vacancies}. NBER Working Paper No. 28961.
\item Arntz, Melanie, Gregory, Terry, and Zierahn, Ulrich (2016). \textit{The Risk of Automation for Jobs in OECD Countries}. OECD Social, Employment and Migration Working Papers, 189.
\end{itemize}

\end{document}
