\documentclass[12pt]{article}
\usepackage[a4paper,margin=1in]{geometry}
\usepackage{titlesec}
\usepackage{lmodern}
\usepackage[utf8]{inputenc}
\usepackage[T1]{fontenc}
\usepackage[dvipsnames]{xcolor}
\usepackage{colortbl} % necesar pentru \rowcolors
\usepackage{booktabs}
\usepackage{array}
\usepackage{longtable}
\usepackage{fancyhdr}
\usepackage[numbers]{natbib}
\usepackage{hyperref}
\hypersetup{
    colorlinks=true,
    linkcolor=blue,
    citecolor=blue,
    urlcolor=blue
}

\setlength{\headheight}{14.5pt}
\pagestyle{fancy}
\fancyhf{}
\rhead{Page \thepage}

\title{
    \textbf{Hybrid Work Contracts in the Age of Artificial Intelligence}\\[1ex]
    \large \normalfont A legal and economic transition model
}
\author{Teodor Berger}
\date{May 2025}

\begin{document}
\maketitle

\begin{abstract}
This academic proposal introduces a hybrid work model designed to protect workers from displacement due to AI integration. It outlines a phased reduction in workload while preserving employment status and salary rights, and proposes a legal mechanism obligating governments and AI companies to compensate employees affected prior to the model's adoption.
\end{abstract}

\section{Introduction}
The emergence of generative artificial intelligence tools has rapidly transformed the global labor market. Without adequate preparation or regulatory frameworks, millions of employees risked losing their jobs before any new legal protection models were adopted. This proposal presents a legally binding hybrid work contract system that governments must implement to prevent social and economic instability.

\section{General Principles}
\begin{enumerate}
    \item \textbf{No Unjust Dismissals:} Employees affected by AI are not allowed to be terminated unless they violate professional ethics or commit acts of misconduct.
    \item \textbf{Hybrid Contract:} Employees can transition to a hybrid contract offering between 1 and 5 working days per week.
    \item \textbf{Proportional Pay:} Base salary is reduced according to working time, with compensatory bonuses.
    \item \textbf{Income Protection:} An additional 10\% of salary is covered by the state.
    \item \textbf{Age Compensation:} Employees over 50 receive additional pay bonuses based on age.
\end{enumerate}

\section{Work Time and Salary Structure}

\begin{table}[ht]
\centering
\rowcolors{1}{blue!10}{white}
\begin{tabular}{|c|c|c|c|}
\hline
\textbf{Days/Week} & \textbf{Total Hours} & \textbf{Base Salary (\%)} & \textbf{Total Salary (\%)} \\ \hline
1 & 3 & 71 & 81 \\ \hline
2 & 6 & 73 & 83 \\ \hline
3 & 9 & 74 & 84 \\ \hline
4 & 12 & 75 & 85 \\ \hline
5 & 15 & 75 & 85 \\ \hline
\end{tabular}
\caption{Base salary and total salary (including 10\% state support)}
\end{table}

\section{Age Bonus Policy}
\begin{itemize}
    \item Age 50–55: +4\% salary bonus
    \item Age 56–60: +7\% salary bonus
    \item Age 61+: +10\% salary bonus
\end{itemize}

\section{Case Studies}

\subsection{Maria (Age 25) – Nurse}
Maria chooses to work 1 day per week (3 hours total).
\begin{itemize}
    \item Base salary: 71\%
    \item State support: +10\%
    \item Total: \textbf{81\%}
\end{itemize}

\subsection{Andrei (Age 57) – Factory Worker}
Andrei works 4 days per week (12 hours total).
\begin{itemize}
    \item Base salary: 75\%
    \item State support: +10\%
    \item Age bonus: +7\%
    \item Total: \textbf{92\%}
\end{itemize}

\section{Legal Liability for Early Displacements}
If companies or AI developers (e.g., OpenAI, xAI) introduced models that implicitly replaced human labor without notifying governments in advance, the responsibility lies with both the company and the state. Affected workers must be:
\begin{itemize}
    \item Financially compensated for moral and economic damages.
    \item Rehired into their previous role or offered an equivalent job.
\end{itemize}

\section{Implementation Recommendations}
\begin{itemize}
    \item Apply the model in AI-impacted sectors: administration, health, education, industry.
    \item Gradually extend to other vulnerable sectors.
    \item Enforce contracts via national labor codes.
\end{itemize}

\section{Conclusion}
Hybrid work contracts offer a realistic, fair, and scalable solution to mass job displacement. With proportional working time and guaranteed protections, both companies and employees benefit from a smooth, dignified transition into the AI era.

\begin{thebibliography}{9}

\bibitem{brynjolfsson2023ai}
Brynjolfsson, E., \& McAfee, A. (2023).
\textit{Artificial Intelligence and the Future of Work}.
Journal of Economic Perspectives, 37(2), 25–48.

\bibitem{acemoglu2021ai}
Acemoglu, D., \& Restrepo, P. (2021).
\textit{AI and Jobs: Evidence from Online Vacancies}.
NBER Working Paper No. 28961.

\bibitem{arntz2016risk}
Arntz, M., Gregory, T., \& Zierahn, U. (2016).
\textit{The Risk of Automation for Jobs in OECD Countries}.
OECD Social, Employment and Migration Working Papers, 189.

\end{thebibliography}

\end{document}
