\documentclass[12pt]{article}
\usepackage[margin=1in]{geometry}
\usepackage{array}
\usepackage{colortbl}
\usepackage{xcolor}
\usepackage{booktabs}
\usepackage{longtable}
\usepackage{fancyhdr}
\usepackage{lipsum}
\usepackage{titlesec}
\usepackage{titling}
\usepackage{natbib}
\usepackage{graphicx}
\usepackage{float}

\setlength{\headheight}{14.5pt}
\pagestyle{fancy}
\fancyhf{}
\rhead{\thepage}

\titleformat{\section}{\large\bfseries}{\thesection}{1em}{}
\titleformat{\subsection}{\normalsize\bfseries}{\thesubsection}{1em}{}

\title{\large A Legal and Economic Transition Model for AI Integration in the Workforce}
\author{Teodor Berger}
\date{May 20, 2025}

\begin{document}

\maketitle

\section*{Abstract}
This academic proposal outlines a structured legal and economic framework designed to mitigate mass layoffs caused by AI integration across global labor markets. It proposes a flexible hybrid contract model to preserve human employment and dignity, avoid economic collapse, and ensure corporate, governmental, and ethical responsibility.

\section{Introduction}
The accelerated implementation of AI systems across industries has resulted in the displacement of countless workers before any legal or transitional protections were in place. This proposal introduces a Hybrid Employment Contract Model (HECM), which must be implemented immediately to prevent further harm and to restore justice for employees already affected.

If governments fail to regulate AI deployment or do not offer protective measures from the start, affected employees must be financially compensated and re-employed by state obligation or assigned an equivalent public-sector position.

\section{Core Principles}

\subsection{1. No Forced Dismissals}
Employees cannot be dismissed solely because of AI implementation. Exceptions apply only in cases of:
\begin{itemize}
  \item Serious misconduct or legal violations.
  \item Voluntary resignation.
  \item Documented incapacity to fulfill the adjusted role.
\end{itemize}

\subsection{2. Flexible Contract Options}
Each employee can select a hybrid model adapted to their availability:
\begin{itemize}
  \item 1–5 working days per week.
  \item Fixed hourly workload: 3 hours/day.
  \item Weekly cap: 15 hours.
\end{itemize}

\section{Revised Compensation Structure}
Employees earn a scaled salary based on number of days worked, combined with state-sponsored compensation and age-related bonuses.

\begin{table}[H]
\centering
\rowcolors{1}{blue!10}{white}
\begin{tabular}{>{\raggedright}p{3cm} >{\centering}p{2.5cm} >{\centering}p{2.5cm} >{\centering}p{2.5cm} >{\centering\arraybackslash}p{2.5cm}}
\toprule
\textbf{Workdays/week} & \textbf{Weekly Hours} & \textbf{Base Salary (\%)} & \textbf{State Support (\%)} & \textbf{Total (\%)} \\
\midrule
1 day  & 3h   & 55\% & 10\% & 65\% \\
2 days & 6h   & 60\% & 10\% & 70\% \\
3 days & 9h   & 65\% & 10\% & 75\% \\
4 days & 12h  & 70\% & 10\% & 80\% \\
5 days & 15h  & 75\% & 10\% & 85\% \\
\bottomrule
\end{tabular}
\caption{Hybrid Work Compensation Table (Without Age Bonus)}
\end{table}

\subsection{Age Bonus (Cumulative)}
\begin{itemize}
  \item Age 50–55: +4\%
  \item Age 56–60: +7\%
  \item Age 61+: +10\%
\end{itemize}

\section{Case Studies}

\subsection{Maria (Age 25) – Nurse}
Maria chooses a 3-day/week contract (9h total). She receives:
\begin{itemize}
  \item Base salary: 65\%
  \item State support: 10\%
  \item Age bonus: 0\%
  \item \textbf{Total: 75\%}
\end{itemize}

\subsection{Andrei (Age 57) – Factory Worker}
Andrei works four days a week (12h total). He receives:
\begin{itemize}
  \item Base salary: 70\%
  \item State support: 10\%
  \item Age bonus: 7\%
  \item \textbf{Total: 87\%}
\end{itemize}

\section{Legal Obligations}
Any AI developer (e.g., OpenAI, xAI) must notify the state prior to deployment of workforce-displacing technologies. Governments must:
\begin{itemize}
  \item Provide immediate reemployment options.
  \item Compensate affected employees if no prior legal protections were implemented.
\end{itemize}

\section{Conclusion}
This policy proposes a just transition mechanism for workers displaced by AI. It ensures fairness, economic continuity, and shared responsibility. Failure to implement will result in further inequality and state-level accountability.

\section*{References}
\begin{itemize}
  \item Brynjolfsson, E., \& McAfee, A. (2023). Artificial Intelligence and the Future of Work. \textit{Journal of Economic Perspectives, 37}(2), 25–48.
  \item Acemoglu, D., \& Restrepo, P. (2021). AI and Jobs: Evidence from Online Vacancies. \textit{NBER Working Paper No. 28961}.
  \item Arntz, M., Gregory, T., \& Zierahn, U. (2016). The Risk of Automation for Jobs in OECD Countries. \textit{OECD Social, Employment and Migration Working Papers, 189}.
\end{itemize}

\end{document}
