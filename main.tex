\documentclass[12pt]{article}
\usepackage[a4paper,margin=1in]{geometry}
\usepackage{titlesec}
\usepackage{fancyhdr}
\usepackage{graphicx}
\usepackage{booktabs}
\usepackage{array}
\usepackage[table,xcdraw]{xcolor}
\usepackage{lipsum}
\usepackage{natbib}
\usepackage{hyperref}

\setlength{\headheight}{14.5pt}
\pagestyle{fancy}
\fancyhf{}
\rfoot{\thepage}

\titleformat{\title}{\normalfont\bfseries\LARGE}{\thesection}{1em}{}
\titleformat{\section}{\normalfont\large\bfseries}{\thesection}{1em}{}
\titleformat{\subsection}{\normalfont\normalsize\bfseries}{\thesubsection}{1em}{}

\title{\textbf{Hybrid Work Contracts in the Age of Artificial Intelligence}\\[0.5em]
\large A legal and economic transition model}
\author{Teodor Berger}
\date{May 2025}

\begin{document}
\sloppy
\maketitle

\section*{Abstract}
This project proposes a legal and economic model to support hybrid work contracts in the age of artificial intelligence. The model addresses job security, fair income distribution, and state responsibility in protecting employees affected by AI integration. It aims to bridge the transitional gap caused by the sudden automation of human labor before governments had time to legislate or adapt.

\section{Introduction}
The advent of AI in the labor market has triggered widespread replacement of human jobs. Governments and corporations failed to anticipate the social consequences, leading to premature layoffs. This model aims to establish protective measures for employees who were dismissed before the introduction of hybrid contracts involving both AI systems and human workers.

\section{Objectives}
\begin{itemize}
  \item Prevent unjust layoffs post-AI integration.
  \item Establish fair salary percentages for hybrid contracts.
  \item Ensure state participation in income support.
  \item Recognize age-based bonuses as compensation for long-term service.
\end{itemize}

\section{Model Structure}

\subsection{Legal Principles}
\begin{enumerate}
  \item \textbf{No Unjust Dismissals:} Employees dismissed before AI integration must be protected or compensated unless layoffs were due to misconduct or company bankruptcy.
  \item \textbf{State Responsibility:} The state shares 10\% of the salary cost in hybrid contracts to stabilize income.
  \item \textbf{Contractual Duty:} AI cannot be a legal substitute for contractual obligations without renegotiation or consent.
\end{enumerate}

\subsection{Weekly Structure and Compensation}

\rowcolors{1}{blue!10}{white}
\begin{table}[h!]
\centering
\begin{tabular}{>{\bfseries}c c c c c}
Workdays/week & Weekly Hours & Base Salary (\%) & State Support (\%) & Total (\%) \\
\midrule
1 & 3 h & 55 & 10 & 65 \\
2 & 6 h & 60 & 10 & 70 \\
3 & 9 h & 65 & 10 & 75 \\
4 & 12 h & 70 & 10 & 80 \\
5 & 15 h & 75 & 10 & 85 \\
\end{tabular}
\caption{Hybrid Work Compensation Model}
\end{table}

\subsection{Age-Based Bonuses}

\begin{table}[h!]
\centering
\begin{tabular}{>{\bfseries}c c}
Age Range & Bonus (\%) \\
\midrule
50--55 & 4 \\
56--60 & 7 \\
61+ & 10 \\
\end{tabular}
\caption{Seniority Bonus}
\end{table}

\subsection{Example Case: Andrei, Age 57}

\begin{itemize}
  \item Work: 4 days/week, 12h total
  \item Base salary: 70\%
  \item State support: +10\%
  \item Age bonus: +7\%
  \item \textbf{Total: 87\%}
\end{itemize}

\section{Conclusion}
This model defends labor rights in the AI age. Those who were dismissed before governments regulated AI integration must be compensated or reinstated via protected hybrid contracts. Technology should not overwrite dignity, fairness, or human contribution.

\section*{References}

\begin{itemize}
\item Brynjolfsson, Erik and McAfee, Andrew (2023). \textit{Artificial Intelligence and the Future of Work}. Journal of Economic Perspectives, 37(2), 25--48.
\item Acemoglu, Daron and Restrepo, Pascual (2021). \textit{AI and Jobs: Evidence from Online Vacancies}. NBER Working Paper No. 28961.
\item Arntz, Melanie, Gregory, Terry, and Zierahn, Ulrich (2016). \textit{The Risk of Automation for Jobs in OECD Countries}. OECD Social, Employment and Migration Working Papers, 189.
\end{itemize}

\end{document}
